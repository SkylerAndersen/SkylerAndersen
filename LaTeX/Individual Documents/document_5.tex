\documentclass[12pt]{article}
\parindent 0px
\pagestyle{empty}
\usepackage{amssymb,amsfonts,amsmath,float}
\begin{document}
\begin{center}
Document 5: Linear Transformations
\end{center}

\setlength{\leftskip}{0 in}
$\,\,\,$ We define a test used to see if a transformation is linear. For the definition of linearity and test, we use two vectors that have the following definitions.
$$\text{Let } k \text{ be an arbitrary scalar. Set } \vec{v}=\begin{bmatrix}v_1\\v_2\end{bmatrix},\vec{w}=\begin{bmatrix}w_1\\w_2\end{bmatrix}
$$

Definition: A transformation $T:\mathbb{R}^n\rightarrow \mathbb{R}^m$ is linear if and only if.
\begin{enumerate}
\item $T(\vec{v}+\vec{w})=T(\vec{v})+T(\vec{w})$.
\item $T(k\vec{v})=kT(\vec{v})$.\\
\end{enumerate}

$\,\,\,$Let $m,n\in\mathbb{Z}$ be arbitrary. Define $T:\mathbb{R}^m\rightarrow\mathbb{R}^n$. If the transformation is linear, then there must be a matrix, $A$, called the standard matrix, such that $T(\vec{x})=A\vec{x}$.

\end{document}