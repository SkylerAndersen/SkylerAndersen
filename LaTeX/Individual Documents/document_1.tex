\documentclass[12pt]{article}
\usepackage{amsfonts,amsmath,amssymb,float}
\pagestyle{empty}
\parindent 0px
\begin{document}
\begin{center}
Document 1: Linear Equations and General Systems
\end{center}

\setlength{\leftskip}{0.25 in}
The general linear system looks like:

$$a_{11}x_1+a_{12}x_{2}+\cdots+a_{1n}x_n=b_1$$
\setlength{\leftskip}{1.3 in}
$\vdots$
$$a_{m1}x_1+a_{m2}x_{2}+\cdots+a_{mn}x_n=b_n$$

\setlength{\leftskip}{0 in}

$\,\,\,$ These would have $m$ equations with $n$ unknowns each. Any system of linear equations can have only three types of solutions: 0 solutions, 1 solution, or $\infty$ solutions.

$\,\,\,$ For cases in $2D$, we visualize the solutions as overlapping for infinite solutions, parallel for no solution, and intersecting for a unique solution. In $3D$, three intersecting planes represent infinite solutions, no universal intersection point represents no solution, and one solution would be represented by planes all at 90 degrees, intersecting at one place (i.e. a plane in xy-space, one in yz-space, and one in xz-space).\\

$\,\,\,$ Elementary Operations Theorem: This theorem provides a system for solving a general system of linear equations.

\begin{enumerate}
\item Interchange two equations: $E_i \leftrightarrow E_j$
\item Multiplication by a scalar: $E_i \leftrightarrow sE_i$
\item Addition of a constant multiple of one equation to another: $E_i \leftrightarrow E_i + kE_j$
\end{enumerate}

$\,\,\,$ This theorem gives way to early concepts such as solving a linear system of equations by elimination. For more generalized algorithmic solutions, we use matrices. An example of a system of equations and its matrix representations is shown below.

$$\left.
\begin{matrix}
x_2+x_3=4\\
x_1-x_2+2x_3=1\\
2x_1+x_2-x_3=6
\end{matrix}
\right|
A=
\begin{bmatrix}
0 & 1 & 1\\
1 & -1 & 2\\
2 & 1 & -1
\end{bmatrix}
,
X=
\begin{bmatrix}
x_1\\
x_2\\
x_3
\end{bmatrix}
,
b=
\begin{bmatrix}
4\\
1\\
6
\end{bmatrix}
$$
\end{document}