\documentclass[11 pt]{article}
\usepackage{amsfonts,amssymb,amsmath}
\parindent 0px
\pagestyle{empty}
\begin{document}
\begin{center}
Document 18: Change of Bases\\
\end{center}

\setlength{\leftskip}{0 in}
$\,\,\,$ In this section, we are investigating changing the basis of a vector subspace. Consider the following example.

\begin{center}
Let $\mathbb{B}_1=\lbrace f_1,f_2,\cdots,f_n\rbrace,\mathbb{B}_2=\lbrace g_1,g_2,\cdots,g_n\rbrace$.

$f=c_1f_1+c_2f_2+\cdots+c_nf_n$

$[f]_{\mathbb{B}_1}=\begin{bmatrix}c_1\\c_2\\\vdots\\c_n\end{bmatrix}$.
\end{center}

$\,\,\,$Then, putting the coordinates into basis 2 yields:
\begin{center}
$[f]_{\mathbb{B}_2}=[c_1f_1+c_2f_2+\cdots+c_nf_n]_{\mathbb{B}_2}$.

$[f]_{\mathbb{B}_2}=c_1[f_1]_{\mathbb{B}_2}+c_2[f_2]_{\mathbb{B}_2}+\cdots+c_n[f_n]_{\mathbb{B}_2}$.

$[f]_{\mathbb{B}_2}=\begin{bmatrix}[f_1]_{\mathbb{B}_2} & [f_2]_{\mathbb{B}_2} & \cdots & [f_n]_{\mathbb{B}_2}\end{bmatrix}\begin{bmatrix}c_1\\c_2\\\vdots\\c_n\end{bmatrix}$
$[f]_{\mathbb{B}_2}=\begin{bmatrix}[f_1]_{\mathbb{B}_2} & [f_2]_{\mathbb{B}_2} & \cdots & [f_n]_{\mathbb{B}_2}\end{bmatrix}[f]_{\mathbb{B}_1}$
\end{center}

$\,\,\,$We define the change of basis transformation in matrix form from $[f]_{\mathbb{B}_1}$ to $[f]_{\mathbb{B}_2}$ with $S=\begin{bmatrix}[f_1]_{\mathbb{B}_2} & [f_2]_{\mathbb{B}_2} & \cdots & [f_n]_{\mathbb{B}_2}\end{bmatrix}$.

$\,\,\,$Let $D$ be an the standard matrix of a given a transformation $T:V\rightarrow V,T(f)=D(f)=f\prime$. Let $f\in V=\text{span}\lbrace e^x,e^{-x}\rbrace$. We want to find the standard matrix for the transformations in two different Bases. We know that $\mathbb{B}_2\lbrace e^x,e^{-x}\rbrace$ and $\mathbb{B}_1\lbrace e^x+e^{-x},e^x-e^{-x}\rbrace$.\\

The steps to finding these matricies are always the same.
\begin{enumerate}
\item Apply the transformation to each of the bases elements.
\item Rewrite each output in coordinates.
\item Define column matrices in terms of the coefficients of the coordinates.
\item The solution matrix is a block matrix made up of these column vectors in order.
\end{enumerate}

$\,\,\,$Lets return to our example with basis $\mathbb{B}_1$ and basis $\mathbb{B}_2$ for the transformation $T(f)=f\prime$.\\

For basis one:

$\left.\begin{matrix}T(e^x)=e^x=1(e^x)+0(e^{-x})\vdash \begin{bmatrix}1\\0\end{bmatrix}\\T(e^{-x})=-e^{-x}=0(e^x)+(-1)(e^{-x})\vdash \begin{bmatrix}0\\-1\end{bmatrix}\end{matrix}\right\rbrace$
$\therefore A=\begin{bmatrix}1 & 0\\0 & -1\end{bmatrix}$\\

For basis two:

$\left.\begin{matrix}T(e^x+e^{-x})=e^x-e^{-x}=0(e^x+e^{-x})+1(e^x-e^{-x})\vdash \begin{bmatrix}0\\1\end{bmatrix}\\T(e^x-e^{-x})=e^x+e^{-x}=1(e^x+e^{-x})+0(e^x-e^{-x})\vdash \begin{bmatrix}1\\0\end{bmatrix}\end{matrix}\right\rbrace$
$\therefore B=\begin{bmatrix}0 & 1\\1 & 0\end{bmatrix}$
\end{document}