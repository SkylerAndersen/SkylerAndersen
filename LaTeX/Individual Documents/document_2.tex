\documentclass[12pt]{article}
\usepackage{float,amsfonts,amssymb,amsmath}
\pagestyle{empty}
\parindent 0px
\begin{document}
\begin{center}
Document 2: Solving Linear Systems
\end{center}

\setlength{\leftskip}{0in}
$\,\,\,$ Gauss-Jordan Elimination: This is a strategy for solving linear equations. One example of Guass or Gauss elimination is below.

Solve the following system:
$
\left\lbrace
\begin{matrix}
x_2+x_3=4\\
x_1-x_2+2x_3=1\\
2x_1+x_2-x_3=6
\end{matrix}
\right.
$\\\\

$\left[\left.\begin{matrix}
0 & 1 & 1\\
1 & -1 & 2\\
2 & 1 & -1
\end{matrix}
\right|
\begin{matrix}
4\\
1\\
6
\end{matrix}
\right]
$
$\rightarrow$
$\left[\left.\begin{matrix}
2 & 1 & -1\\
1 & -1 & 2\\
0 & 1 & 1
\end{matrix}
\right|
\begin{matrix}
6\\
1\\
4
\end{matrix}
\right]
\rightarrow$
$\left[\left.\begin{matrix}
2 & 1 & -1\\
0 & -\frac{3}{2} & \frac{5}{2}\\
0 & 1 & 1
\end{matrix}
\right|
\begin{matrix}
6\\
-2\\
4
\end{matrix}
\right]
\rightarrow$
$\left[\left.\begin{matrix}
2 & 1 & -1\\
0 & -3 & 5\\
0 & 1 & 1
\end{matrix}
\right|
\begin{matrix}
6\\
-4\\
4
\end{matrix}
\right]
\rightarrow$\\\\

$\left[\left.\begin{matrix}
2 & 1 & -1\\
0 & -3 & 5\\
0 & 0 & \frac{8}{3}
\end{matrix}
\right|
\begin{matrix}
6\\
-4\\
\frac{8}{3}
\end{matrix}
\right]
$\\\\

$\,\,\,$ Hence, the system is transformed into a new system.

$$\begin{matrix}
2x_1+x_2-x_3=6\\
-3x_2+5x_3=-4\\
8x_3=8
\end{matrix}$$

$\,\,\,$ This system is much easier to solve, but it can be reduced further. Guass-Jordan elimination leaves only numbers along the top-left to bottom-right diagonal. When multiplying rows, we often choose a reference point we call the pivot.

$\,\,\,$ Given a different set of equations, lets solve the system with Gauss-Jordan elimination. We will reduce the matrix to echelon form.

$\,\,\,$ Let the system be $\left\lbrace\begin{matrix}x_1+x_2+x_3+4x_4=4\\2x_1+3x_2+4x_3+9x_4=16\\-2x_1+3x_3-7x_4=11\end{matrix}\right.$.\\

$\,\,\,$ We rewrite the system in matrix form and transform it to reduced row echelon form by performing operations on each row, or adding rows to other rows.

$$\left[\left.\begin{matrix}
1 & 1 & 1 & 4\\
2 & 3 & 4 & 9\\
-2 & 0 & 3 & -7
\end{matrix}\right|
\begin{matrix}
4\\16\\11
\end{matrix}\right]
\rightarrow
\left[\left.\begin{matrix}
1 & 1 & 1 & 4\\0 & 1 & 2 & 1\\0 & 2 & 5 & 1
\end{matrix}\right|
\begin{matrix}
4\\8\\19
\end{matrix}\right]
\rightarrow
\left[\left.\begin{matrix}
1 & 0 & -1 & 3\\0 & 1 & 2 & 1\\0 & 0 & 1 & -1
\end{matrix}\right|
\begin{matrix}
-4\\8\\3
\end{matrix}\right]
\rightarrow
$$$$
\left[\left.\begin{matrix}
1 & 0 & 0 & 2\\0 & 1 & 0 & 3\\0 & 0 & 1 & -1
\end{matrix}\right|
\begin{matrix}
-1\\2\\3
\end{matrix}\right]
$$\\

$\,\,\,$ After reducing this matrix to echelon form, we have the following system of equations, where $x_4$ is considered a “free parameter".
$$\left\lbrace\begin{matrix}
x_1+2x_4=-1\\x_2+3x_4=2\\x_3-x_4=3
\end{matrix}\right.=\left\lbrace\begin{matrix}
x_1=-1-2x_4\\x_2=2-3x_4\\x_3=3+x_4\\x_4=x_4
\end{matrix}\right.$$
$\,\,\,$ In cases of a free parameter, we often use the variable $t$; hence, here we define $t=x_4$, and redefine our solution set using the following matrices.
$$\begin{bmatrix}
x_1\\x_2\\x_3\\x_4
\end{bmatrix}
=
\begin{bmatrix}
-1\\2\\3\\0
\end{bmatrix}
+t
\begin{bmatrix}
-2\\-3\\1\\1
\end{bmatrix}$$

$\,\,\,$ This system has infinite solutions. A system that has no solution would have a row of all zeros, with a non-zero number on the right hand side of the vertical line. Below are some useful terms to describe behaviors we have see while solving the systems.\\

Definition: an $m$ x $n$ matrix is in Row Echelon Form if

\begin{enumerate}
\item All zero rows appear at the bottom.
\item If a row has nonzero entries, then the first nonzero entry is 1 (This is called the leading one).
\item If a row contains a leading one, then each row above contains a leading one further to the left.
\end{enumerate}

Definition: an $m$ x $n$ matrix is in Reduced Row Echelon Form if

\begin{enumerate}
\item All zero rows appear at the bottom.
\item If a row has nonzero entries, then the first nonzero entry is 1 (This is called the leading one).
\item If a row contains a leading one, then each row above contains a leading one further to the left.
\item If a column contains a leading one, then all other entries in that column are zero.
\end{enumerate}
\end{document}