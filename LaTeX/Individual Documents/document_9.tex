\documentclass[11pt]{article}
\usepackage{amsfonts,amssymb,amsmath,float}
\parindent 0px
\pagestyle{empty}
\begin{document}
\begin{center}
Document 9: Computing the inverse of a matrix
\end{center}

\setlength{\leftskip}{0in}
$\,\,\,$ To compute the inverse of a matrix, write the augmented matrix $[A_{n\times n}|I_n]$, compute the reduced row echelon form of the augmented matrix, yielding the identity matrix on the right. In other terms, rref$([A|I])=[I|A^{-1}]$.

$\,\,\,$ For the special case of a matrix $A_{2\times 2}=\begin{bmatrix}a & b\\c & d\end{bmatrix}$, $A$ is invertible $\Leftrightarrow$ det$(A)=ad-bc\not =0$. Let $A_{p\times n},B_{m\times p}$. Then $BA$ is defined and $T(\vec{x})=B(A\vec{x}),T:\mathbb{R}^n\rightarrow\mathbb{R}^m$ and $(BA)_{m\times n}$. In general, $BA\not =AB$ unless $A=B^{-1}$ and $B=A^{-1}$\\

Definition: Matrix properties
\begin{enumerate}
\item Let $A_{q\times n},B_{m\times p}$ be arbitrary matrices. Then $BA$ is defined $\Leftrightarrow p=q$.
\item Let $A_{p\times n},B_{m\times p}$ be arbitrary matrices. Then $BA$ is defined, and $BA_{m\times n}$ is the standard matrix for a transformation $T:\mathbb{R}^n\rightarrow\mathbb{R}^m,T(\vec{x})=B(A\vec{x})$
\item For arbitrary matrices $A,B$ it is not guaranteed that $AB=BA$.\\
\end{enumerate}

Properties of Invertable Matrices:
\begin{enumerate}
\item Let $A_{n\times n}$ be an invertable matrix. Then, $AA^{-1}=I_n,A^{-1}A=I_n$.
\item Let $A_{m\times n}$ be an invertable matrix. Then, $I_mA_{m\times n}=A_{m\times n}I_n=A$.
\item Matrix multiplication is associative, so $(AB)C=A(BC)=ABC$.
\item Let $A_{n\times n},B_{n\times n}$ be invertable matrices. Then $(AB)^{-1}=A^{-1}B^{-1}$.
\item Let $A_{m\times p},B_{m\times p},C_{p\times n},D_{p\times n}$ be matrices. Then $A(C+D)=AC+AD$ and $(A+B)C=AC+BC$\\
\end{enumerate}

Criteria for Invertability: Let $A_{n\times n},B_{n\times n}$. Set $BA=I_n$. Then,
\begin{enumerate}
\item $A$ and $B$ are both invertible.
\item $A^{-1}=B$, and $B^{-1}=A$
\item $AB=I_n$
\end{enumerate}
\end{document}